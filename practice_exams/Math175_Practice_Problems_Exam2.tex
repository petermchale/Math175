\documentclass[11pt]{article}
\setlength{\oddsidemargin}{0in}
\setlength{\evensidemargin}{0in}
\setlength{\textwidth}{6.5in}
\setlength{\parindent}{0in}
\setlength{\parskip}{\baselineskip}

\usepackage{amsmath}
\usepackage{amsthm}
\usepackage{amssymb}
\usepackage[dvips]{graphics}
\usepackage{epsfig}
\usepackage{color}
\usepackage{fullpage}
\newcommand{\F}{{\mathbb F}}


\newtheorem{theorem}{Theorem}
\newtheorem*{lemma}{Lemma}
\newtheorem*{prop}{Proposition}
\newtheorem*{cor}{Corollary}
\newtheorem*{theo}{Theorem}

\begin{document}
\begin{center}
{\Large \bf Math 175: Combinatorics} \\
{\large \bf Exam 2 Practice Problems}
\end{center}

Since Homework 5 was posted we continued our discussion of inclusion-exclusion.  We gave a proof similar to the one given for Theorem 2.6 of HHM (but our proof was more detailed).  

We then started to discuss derangements and the $n$ students, $n$ lunchboxes problem.  Derangements are covered in pages 160-161 of HHM.  

We spent some time discussing the set of all permutations of $\{1,2,\ldots, n\}$, which we denoted by $S_n$.  We saw a few ways of writing a permutation: two line notation, one line notation, and cycle notation. Section 2.7.1 of HHM is on `Permutation Groups' has a nice explanation of these concepts.  This section contains some additional material that we will not need in this course (for example, you do not need to know what a group is).

On Monday we will finish up the proof of our formula for $D(n)$, the number of derangements of $\{1,2,\ldots, n\}$.  We will then discuss the number of permutations with a given cycle structure.  After that we will return to the problem of 100 prisoners who need to find their names in 100 boxes that we discussed in the first lecture.
  

\section{True/False}

\begin{enumerate}

\item At a party with $10$ guests there are at least two people who know the same number of other guests.

{\bf Note}: If person A `knows' person B, then person B must also `know' person A.




\item Let $C_k$ denote the $k$th Catalan number.  For $k \ge 1$,
\[
C_k = \frac{2^k}{(k+1)!} \prod_{i=1}^k (2i-1).
\]

\item A professor has been working for the same department for 30 years.  There are two semesters in a year.  She taught two courses in each semester.  The department offers 15 different courses.  There must have been two semesters when this professor taught exactly the same pair of courses.

\end{enumerate}

\section{Problems}


\begin{enumerate}

\item A $k$-digit number cannot have $0$ as its first digit (for example $09$ is not a two-digit number.)  A number is a palindrome if it is the same read forwards and backwards (for example $172271$ is a palindrome, but $2320$ is not.)  \\
How many $2k+1$ digit numbers are not palindromes?
 
\item How many arrangements of MISSISSIPPI do not have consecutive I's?


\item Recall that a \emph{derangement} of $\{1,2,\ldots, n\}$ is a permutation with no fixed points.  Let $D(n)$ denote the number of derangements of $\{1,2,\ldots, n\}$.  By convention $D(0) = 1$.

\begin{enumerate}

\item Directly calculate $D(1), D(2)$, and $D(3)$.

\item Give a combinatorial explanation for the following identity:
\[
n! = \sum_{k=0}^n \binom{n}{k} \cdot D(n-k),
\]
(that is, show that both sides above are counting the same thing.)

\item Use the identity from the previous part to compute $D(4)$.

\end{enumerate}


\item Let $F(n)$ be the number of subsets of the set $\{1,2,\ldots, n\}$ that contain no three consecutive integers.  For example, $F(1) = 2,\ F(2) = 4$, and $F(3) = 7$.  

Find a recurrence satisfied by $F(n)$.

\item For a positive integer $n \ge 0$, show that
\[
\sum_{k=0}^n \frac{1}{k+1} \cdot \binom{n}{k} = \frac{2^{n+1}-1}{n+1}.
\]

{\bf Hint}: You may want to rewrite this equation somehow instead of trying to apply induction right away.

\item Suppose $n \ge 2$.  We choose $n+2$ numbers from the set $\{1,2,\ldots, 3n\}$.  Prove that there are always two among the chosen numbers whose difference is more than $n$ but less than $2n$.  

\item How many six letter words in the English alphabet either begin aa or end zz?

%\item The names of $100$ prisoners are placed in $100$ wooden boxes, one name to a box, and the boxes are lined up on a table in a room.  One by one, the prisoners are led into the room; each may look in at most $99$ boxes, but must leave the room exactly as she found it and is permitted no further communication with the others.

%The prisoners have a chance to plot their strategy in advance, and they are going to need it, because unless \emph{every single prisoner finds her own name} all will subsequently be executed.

%The prisoners huddle together and come up with a few strategies.  One suggests: ``I know what we'll do.  Each one of us will go in there and randomly open $99$ of the boxes.  The odds are good that I'll find my name, so this is probably a decent strategy."  

%A second prisoner suggest: ``I have something a little better.  Let's randomly line ourselves up before going in, and if I am prisoner number $i$ in line, I will open every box except box number $i$''.  

%A third prisoner suggests: ``Why don't we try the natural variation of the thing from lecture (which we all went to before being imprisoned.)  That is, we randomly line ourselves up.  If I am prisoner number $i$ in line, I will open box $i$.  If that box contains the name of prisoner $j$ in line, I will open box $j$ next.  I will continue until I either find my name or I have opened $99$ boxes and I have not found my name''.

%For each of these three strategies write down the probability that the prisoners all find their names.

\end{enumerate}




\end{document}