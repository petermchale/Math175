\documentclass[11pt]{article}
\setlength{\oddsidemargin}{0in}
\setlength{\evensidemargin}{0in}
\setlength{\textwidth}{6.5in}
\setlength{\parindent}{0in}
\setlength{\parskip}{\baselineskip}

\usepackage{amsmath}
\usepackage{amsthm}
\usepackage{amssymb}
\usepackage[dvips]{graphics}
\usepackage{epsfig}
\usepackage{color}
\usepackage{fullpage}
\newcommand{\F}{{\mathbb F}}

\begin{document}
\begin{center}
{\Large \bf Math 175: Combinatorics} \\
{\Large \bf Homework 2: Solutions}\\
\end{center}

\begin{enumerate}

\item {\bf Exercise 2.6.2.5 of HHM}:  Suppose that an unlimited number of jelly beans is available in each of five different colors: red, green, yellow, white, and black.  
\begin{enumerate}
\item How many ways are there to select twenty jelly beans?

{\bf Solution}: We write the number of red jelly beans as $x_1$, the number of green as $x_2$, the number of yellow as $x_3$, the number of white as $x_4$, and the number of black as $x_5$.  The ways of choosing $20$ jellybeans are in bijection with non-negative integer solutions 
\[
x_1 + x_2 + x_3 + x_4 + x_5 = 20,
\]
or equivalently, weak compositions of $20$ into five parts.  By adding one to each part, these are in bijection with compositions of $25$ into $5$ parts.  By the `Stars and Bars' theorem we proved in class, this is $\binom{20+5-1}{5-1}$.


\item How many ways are there to select twenty jelly beans if we must select at least two jelly beans of each color?

{\bf Solution}:
The ways of choosing $20$ jellybeans where we must choose at least $2$ in each color are in bijection with solutions to 
\[
x_1 + x_2 + x_3 + x_4 + x_5 = 20,
\]
where each $x_i \ge 2$.  Subtracting $1$ from each $x_i$ shows that the set of such ways is in bijection with the set of compositions of $15$ into $5$ parts.  We recall from our `Stars and Bars' result that this is $\binom{15-1}{5-1}$.

\end{enumerate}


\item {\bf Exercise 2.6.2.6 of HHM}:
A catering company brings fifty identical hamburgers to a party with twenty guests.  
\begin{enumerate}
\item How many ways can the hamburgers be divided among the guests, if none is left over?

{\bf Solution}: The number of ways of giving out the hamburgers is in bijection with nonnegative integer solutions to 
\[
\sum_{i=1}^{20} x_i = 50,
\]
that is, weak compositions of $50$ into exactly $20$ parts.  These are in bijection with compositions of $70$ into $20$ parts.  There are $\binom{70-1}{20-1}$ of these.


\item How many ways can the hamburgers be divided among the guests, if every guest receives at least one hamburger, and none is left over?

{\bf Solution}:  The only difference between this part and the previous one is that now each $x_i \ge 1$.  So we just need to count compositions of $50$ into $20$ parts.  There are $\binom{50-1}{20-1}$ of these.


\item Repeat these problems if there may be burgers left over.

{\bf Solution}:  There is a nice interpretation here if we now think of there being $21$ people who receive hamburgers now instead of $20$, where Person 21 is `Garbage', the hamburgers that get left over.  We are looking for weak compositions of $50$ into $21$ parts.  There are $\binom{50+21-1}{21-1} = \binom{70}{20}$ of these.


So the ways of handing out $20$ hamburgers, where some can be left over, to $20$ people are in bijection with compositions of $50-20$ into $21$ parts, but where the $21$\textsuperscript{th} person is allowed to get $0$ hamburgers (there which is given by $\binom{50-20-1}{21-1}$.

In the second part, the ways of handing out $50$ hamburgers, where some can be left over, to $20$ people, where each person must receive at least one are in bijection with solutions to
\[
x_1+x_2+\cdots + x_{20} + x_{21} = 50,
\]
where each $x_i \ge 1$ except that $x_{21}$ can be equal to $0$.  These solutions are in bijection with compositions of $51$ into $21$ parts.  There are $\binom{51-1}{21-1} = \binom{50}{20}$ of these.
\end{enumerate}


\item 
\begin{enumerate}
\item How many compositions of $n$ use only parts that are even?  \\

{\bf Solution}: If $n$ is odd, there are zero.  This is because the sum of any number of even numbers is even.  If $n$ is even, by considering small numbers we may guess that the answer is $2^{n/2-1}$. 

We will show that this is correct by giving a bijection between compositions of an even number $n$ into even parts, and compositions of $n/2$.  The bijection just takes each part and divides by two.  Clearly, two compositions into even parts cannot be sent to the same composition of $n/2$.  Also, given a composition of $n/2$ we can see that the composition where we double each part is a composition of $n$ into even parts.  This shows that we have found a bijection.

\item How many compositions of $n$ only use parts that are odd?

{\bf Solution}:  We see that for $n=1$ we get one way, for $n=2$ we get one way, for $n=3$ we get two ways, for $n=4$ we get three ways, and for $n=5$ we get five ways.  At this point we might guess that the answer is the $n$th Fibonacci number $F_n$.

We first ask a more specific question.  How many compositions of $n$ use exactly $k$ odd parts?  If $n$ is even, then the answer is zero when $k$ is odd, and when $n$ is odd, the answer is zero when $k$ is even.  We note that by adding $1$ to each part we get a composition of $n+k$ into exactly $k$ even parts.  It is easy to verify that this is a bijection.  Using the reasoning in part (a), dividing each of these parts by two gives a bijection with compositions of $\frac{n+k}{2}$ into exactly $k$ parts.  There are $\binom{\frac{n+k}{2} -1}{k-1}$. 

We suppose that $n$ is odd.  The argument for $n$ even is almost exactly the same.  What are the possible values of $k$?  It has to be of the same parity (even or odd) as $n$, and can be anything less than or equal to $n$.  This gives a sum of binomial coefficients:
\[
\binom{n-1}{n-1} + \binom{n-2}{n-3} + \cdots + \binom{\frac{n+1}{2}}{1}.
\]
Noting that $\binom{n}{k} = \binom{n}{n-k}$ this is equal to
\[
\binom{n-1}{0} + \binom{n-2}{1} + \cdots + \binom{\frac{n+1}{2}}{\frac{n+1}{2} - 1}.
\]
Applying a Fibonacci identity, this is equal to $F_n$.  

{\bf Solution 2}:  Let $f(n)$ count the number of compositions of $n$ into odd parts. We note that $f(1) = f(2) = 1$.  We will show that this sequence satisfies the same recurrence as the Fibonacci sequence.  This will show that for all $n$, the answer is $F_n$.

Suppose $n \ge 2$.  We give a bijection from the set of compositions of $n$ into odd parts and the set of compositions of $n-1$ or $n-2$ into odd parts.  Given a composition of $n$ into odd parts that ends in $1$, delete the last $1$.  This gives a composition of $n-1$ into odd parts and it is clear that this is a one-to-one correspondence.  Given a composition of $n$ into odd parts that ends in something larger than $1$, subtract two from the last part.  This gives a composition of $n-2$ into odd parts.  This is also a one-to-one correspondence.  Taking these two maps together gives our bijection and shows that $f(n) = f(n-1) + f(n-2)$ for $n \ge 2$.  This completes the proof.


\item How many weak compositions of $n$ into exactly $k$ parts use only parts that are even?

{\bf Solution}:  When $n$ is odd, the answer is zero, so we suppose $n$ is even.

We give a bijection between the set of weak compositions of $n$ into exactly $k$ parts using only even parts, and compositions of $n+2k$ using only even parts.  Using the reasoning above the set of such compositions is in bijection with the set of compositions of $\frac{n}{2} + k$ into exactly $k$ parts.  There are $\binom{\frac{n}{2} + k -1}{k-1}$ of these.

Given a weak composition of $n$ into exactly $k$ parts using only even parts, we first add two to every part, then we divide every part by $2$.  The first step gives a bijection with compositions of $n+2k$ into exactly $k$ even parts, and the second step gives a bijection with compositions of $\frac{n+2k}{2}$ into exactly $k$ parts.

\item How many compositions of $n$ into an even number of parts only use parts that are even?  

{\bf Solution}: If $n$ is odd the answer is zero, so we suppose $n$ is even.  We know that there is a bijection between compositions of $n$ using exactly $k$ even parts and compositions of $\frac{n}{2}$ using exactly $k$ parts.  By adding up the relevant binomial coefficients, we know that half of all compositions of $\frac{n}{2}$ have an even number of parts.  There are $2^{\frac{n}{2}-1}$ total compositions of $\frac{n}{2}$, so $2^{\frac{n}{2} -2}$ have an even number of parts.  Our bijection shows that this also counts the number of compositions of $n$ into an even number of even parts.



\end{enumerate}


\item A standard die has six faces labeled $\{1,2,3,4,5,6\}$.  When you roll the die, each face comes up with probability exactly $\frac{1}{6}$.

\begin{enumerate}

\item Suppose you roll a die $k$ times.  What is the probability that you get $k$ distinct numbers? (Your answer should be a function of $k$.)

{\bf Solution}: Rolling a die $k$ times gives a sequence of length $k$ from the alphabet $\{1,2,3,4,5,6\}$.  We want to count the number of such sequences with no repeated letter, divided by the total number of sequences.  The probability is therefore
\[
\frac{\frac{6!}{(6-k)!}}{6^k} = \frac{6!}{(6-k)! 6^k}.
\]



\item Write down a table that shows these values for all $1\le k \le 6$.  What is the smallest value of $k$ for which this is less than $1/2$?

{\bf Note}: You will probably need to use a computer or calculator to compute these values.

{\bf Solution}:  A little computation with a calculator or computer gives the following table:
\[
\begin{tabular}{|c|c|c|c|c|c|c|c|}
\hline
& 1 & 2 & 3 & 4 & 5 & 6 & $k \ge 7$\\
\hline
\text{Probability}& 1 & $\frac{5}{6} $ &$ \frac{5}{9} $ &$\frac{5}{18}$  & $\frac{5}{54}$  &$ \frac{5}{324} $ & 0\\
\hline
\end{tabular}.
\]
We note that $k=4$ is the first time this probability is less than $50\%$.


\end{enumerate}

\item A standard deck of playing cards has $4$ suits (spades $\spadesuit$, diamonds $\diamondsuit$, clubs $\clubsuit$, and hearts $\heartsuit$). In each suit there is exactly one card from each of $13$ denominations ($2,3,4,\ldots, 10$, Jack, Queen, King, and Ace).  This gives $52$ total cards (for example the $2$ of Diamonds and the Jack of Hearts).

Suppose you draw $5$ cards at random.
\begin{enumerate}
\item What is the probability that you do not have two cards of the same denomination?

{\bf Solution}: We first note that there are $\binom{52}{5}$ different possible collections of $5$ cards.  We are asking how many collections have no two cards of the same denomination.  We first count sequences of five cards so that no two have the same denomination.  We have $52$ choices for the first card, $48$ choices for the second card, $44$ choices for the third card, $40$ choices for the fourth card, and $36$ choices for the final card.  To go from sequences to sets, we divide by $5!$ to forget the order in which we choose the cards.  This gives a probability of 
\[
\frac{52\cdot 48\cdot 44\cdot 40\cdot 36}{5! \binom{52}{5}} = \frac{44\cdot 40\cdot 36}{51\cdot 50\cdot 49 \cdot 47} \approx 50.7\%.
\]




\item What is the probability that you have four cards of the same denomination? (This is called a `Four of a Kind'.)

{\bf Solution}:  There are $13$ denominations in which we could have `Four of a Kind'.  Once we choose which of these $13$ collections of four cards we want to have, we need to choose one more card.  There are $48$ choices for this last card.  So the probability it
\[
\frac{\binom{13}{1}\cdot 48}{\binom{52}{5}} = \frac{13\cdot 5\cdot4\cdot 3\cdot 2}{52\cdot 51\cdot 50\cdot 49} \approx 0.00024.
\]


\item What is the probability that you have three, but not four cards of the same denomination? 

{\bf Solution}: We choose which of the $13$ denominations we want to have three cards in.  We then need to choose which $3$ of the four cards in this denomination we would like to have.  Finally, we have two remaining cards to choose, and since neither card can be in our `Three of a Kind' denomination, we have $\binom{48}{2}$ choices for this pair of cards.  The probability is then
\[
\frac{\binom{13}{1}\cdot \binom{4}{3}\cdot \binom{48}{2}}{\binom{52}{5}} \approx 0.02257.
\]

\item What is the probability that you have three cards in one denomination, and your remaining two cards are in the same denomination? (This is called a `Full House'.)

{\bf Solution}: This is a subcase of the previous problem.  The first part is the same- we choose our `Three of a Kind' denomination and then choose which $3$ of the $4$ cards in it we want to have.  We have $48$ remaining cards not in this denomination and want to choose a pair of two from the same denomination.  We first choose which of the $12$ denominations we want to have our pair in, and then choose which $2$ of the $4$ cards in this denomination we want to have.  The probability is then 
\[
\frac{\binom{13}{1}\cdot \binom{4}{3}\cdot \binom{12}{1}\cdot\binom{4}{2}}{\binom{52}{5}} \approx 0.00144.
\]


\end{enumerate}




\item In lecture we saw that once there are $23$ or more people in a room, there is more than a $50\%$ chance of having two people with the same birthday.  (For simplicity we have assumed that there are $365$ possibly birthdays and that they are all equally likely.)  

For $k$ people we can compute this probability exactly.  We line up the $k$ people in some order and ask each of them to write down their birthday.  This gives a word of length $k$ from an alphabet of $365$ letters.  Now we'll consider the birthday problem for triples.  I will break it down into a bunch of steps to try to help guide you through it.
\begin{enumerate}
\item What is the probability that among $k$ people the first two of them have the same birthday, nobody else has that birthday, and every other person has a distinct birthday?

{\bf Solution}: We think of the probability as being given by the following process: We ask $k$ people in some order to say their birthdays.  We count the number of possible sequences of birthday where the first two are the same, nobody else has that birthday, and every other birthday is distinct, and divide by $365^k$, the total number of possible sequences.

We first pick the birthday shared by the first two people.  There are $\binom{365}{1}$ possible days.  We then know that the next $k-2$ people each have distinct birthdays chosen from the $364$ remaining days, giving $\frac{364!}{(364-(k-2))!}$ possible sequences.  This gives the probability as
\[
\frac{\binom{365}{1} \frac{364!}{(364-(k-2))!}}{365^k} = \frac{365!}{(364-(k-2))! 365^k}.
\]

\item What is the probability that among $k$ people exactly two of them have the same birthday?

{\bf Solution}: We first pick the common birthday, which we can do in $\binom{365}{1}$ ways.  We now have to pick the two spots for these people in the sequence of $k$ people.  We can do this in $\binom{k}{2}$ ways.  Now there are $k-2$ distinct birthdays chosen from $364$ possible birthdays and given in some order among the $k-2$ remaining spots.  This gives the probability as 
\[
\frac{\binom{365}{1} \binom{k}{2} \frac{364!}{(364-(k-2))!}}{365^k} = \frac{365! \binom{k}{2}}{(364-(k-2))! 365^k}.
\]

\item What is the probability that among $k$ people, the first two people have the same birthday, the second two people have the same birthday (different from the first pair), and everyone else has a distinct birthday?

{\bf Solution}:  We first choose the birthday shared by the first pair.  There are $365$ ways to do this.  We then choose the birthday shared by the second pair.  There are $364$ ways to do this given the choice we've already made. We then know that there are $k-4$ distinct birthdays chosen from the $363$ remaining days given in some order by the $k-4$ remaining people.  This gives the probability as
\[
\frac{365\cdot 364\cdot \frac{363!}{(363-(k-4))!}}{365^k} = \frac{365!}{ (363-(k-4))! \cdot 365^k}.
\]

\item What is the probability that among $k$ people, exactly two pairs of people share a birthday (and no three people have the same birthday)?

{\bf Solution}: We first pick the two shared birthdays.  We then pick the spots for one of the two pairs, which we can do in $\binom{k}{2}$ ways.  We then pick the spots for the next pair, which we can do in $\binom{k-2}{2}$ ways.  We note that 
\[
\binom{k}{2} \binom{k-2}{2} = \frac{k!}{(k-4)! 2^2}.
\]
We then choose the remaining birthdays in some order in the remaining $k-4$ spots.  This gives 
\[
\frac{\binom{365}{2} \frac{k!}{(k-4)! 2^2} \frac{363!}{(363-(k-4))!}}{365^k} = \frac{365! k!}{2!  (k-4)! 2^2 (363-(k-4))! 365^k}.
\]


\item Show that the probability that among $k$ people there are exactly $m$ pairs of people with the same birthday, and no three people have the same birthday is given by
\[
\left(\frac{365! k!}{m! (365-k+m)! 2^m (k-2m)!}\right) \cdot \frac{1}{365^k}
\]

{\bf Solution}: We adapt the argument give above for the special case $m=2$.  We first choose these $m$ shared birthdays, which we can do in $\binom{365}{m}$ ways.  We then choose the spots for the first pair, followed by the second pair, and so on, down to the $m$th pair.  We can do this in $\binom{k}{2} \binom{k-2}{2}\cdots \binom{k-2(m-1)}{2}$.  We note that this is equal to $\frac{k!}{(k-2m)! 2^m}$.  We now choose the remaining $k-2m$ distinct birthdays in some order from the remaining $365-m$ days, which we can do in $\frac{(365-m)!}{(365-m-(k-2m))!}$ ways.  Note that $\binom{365}{m} (365-m)! = \frac{365!}{m!}$.  This gives the probability as
\[
\frac{365! k!}{(k-2m)! 2^m m! (365+m-k)! 365^k},
\]
completing the proof.

\item Write down a function that gives the probability that among $k$ people, no three of them have the same birthday.  This is just a sum involving the previous part of this problem.

{\bf Solution}: This probability is given as a sum over the number of pairs of people with a common birthday.  This number of pairs can take any value from $0$, in the case when all birthdays are distinct, to $\lfloor \frac{k}{2} \rfloor$.  This gives
\[
\sum_{m=0}^{\lfloor \frac{k}{2} \rfloor} \frac{365! k!}{(k-2m)! 2^m m! (365+m-k)! 365^k}.
\]

\item \textbf{Bonus/Not Required}: Use a computer to evaluate this probability for some values of $k$ (if you can do it without a computer, more power to you).  How many people do we need to have in a room before there is more than a $50\%$ chance that at least three people share a common birthday?

{\bf Solution}: Evaluating in Mathematica with the following commands shows that for $k=87$ the probability is approximately $.500545$ and for $k=88$ the probability is around $.488935$.  Here is the Mathematica code:
\begin{eqnarray*}
P[k\_, m\_] & := & 365!*k!/((k - 2 m)! *2^m*m!*(365 + m - k)!*365^k) \\
PP[k\_] & := & \text{Sum}[P[k, m], \{m, 0, \text{Floor}[k/2]\}] \\
N[PP[88]],
\end{eqnarray*}
which gives $0.488935$.

\end{enumerate}





\end{enumerate}





\end{document}