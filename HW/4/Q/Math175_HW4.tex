\documentclass[11pt]{article}
\setlength{\oddsidemargin}{0in}
\setlength{\evensidemargin}{0in}
\setlength{\textwidth}{6.5in}
\setlength{\parindent}{0in}
\setlength{\parskip}{\baselineskip}

\usepackage{url}

\usepackage{amsmath}
\usepackage{amsthm}
\usepackage{amssymb}
\usepackage[dvips]{graphics}
\usepackage{epsfig}
\usepackage{color}
\usepackage{fullpage}
\newcommand{\F}{{\mathbb F}}

\begin{document}
\begin{center}
{\Large \bf Math 175: Combinatorics} \\
{\Large \bf Homework 4}\\
{\Large Posted Friday Feb 9th}\\
{\Large Due in Discussion Thursday, February 15th}
\end{center}


After the last homework was posted, we continued our discussion of Fibonacci numbers, first counting the number of compositions of $n$ into $1$'s and $2$'s.  We then proved more identities in Pascal's triangle.  We started by showing that $\sum_{k=0}^n \binom{n}{k}^2 = \binom{2n}{n}$.  We first gave a combinatorial proof and then showed how it followed from a more general statement, Vandermonde's Identity.  

We then stated and proved the Binomial Theorem in two different ways, first by induction using the identity $\binom{n}{k} = \binom{n-1}{k-1} + \binom{n-1}{k}$ and then by using a combinatorial argument.  We showed how substituting values of $x$ and $y$ gives new proofs of identities we have already seen, and also leads to new identities.  In particular, we showed how setting $x = i$ and $y = 1$ gave a formula for $\sum_{k=0}^n (-1)^k \binom{2n}{2k}$.  In order to understand this argument, we first reviewed some basic facts about complex numbers.  We also showed how the Binomial Theorem gave another proof of Vandermonde's Identity.

We then talked about anagrams and multinomial coefficients, and stated the Multinomial Theorem and the analogue of the identity $\binom{n}{k} = \binom{n-1}{k-1} + \binom{n-1}{k}$.  Finally, we started to talk more about counting and went over an example involving counting labeled vs. unlabeled teams.

The main topics that we covered are the subjects of Section 3.1 of LVP and Sections 2.2 and 2.3 of HHM.  I mentioned Pascal's Pyramid in lecture: there is a nice picture of the first few levels in Figure 2.1 on page 146 of HHM.

\newpage

\centerline{ \bf \Large Problems}

\begin{enumerate}

\item 
\begin{enumerate}
\item How many subsets of $\{1,2,\ldots, n\}$ contain no two consecutive integers?

\item How many subsets of $\{1,2,\ldots, n\}$ contain no two consecutive integers if $1$ and $n$ also count as consecutive?

\item How many subsets of $\{1,2,\ldots, n\}$ of size $k$ contain no two consecutive integers?

{\bf Hint}: For this part, if you take out $k$ integers from this set, the remaining numbers are grouped into blocks in a natural way.  How many blocks are there and what do we know about their sizes?

\item Write down the identity you get from taking a sum over the result in part (c) and comparing it to the result from part (a).

\end{enumerate}

\item {\bf Exercise 4.3.16 of LVP} 
\begin{enumerate}
\item Prove that every positive integer can be written as a sum of different Fibonacci numbers.

\item Prove even more: every positive integer can be written as the sum of different Fibonacci numbers, so that no two consecutive Fibonacci numbers are used.

\item Show by an example that the representation in (a) is not unique, but also prove that the more restrictive representation in (b) is.
\end{enumerate}

\item In lecture we used the Binomial Theorem to prove that 
\[
\sum_{k=0}^n \binom{2n}{2k} (-1)^k = 2^n \left(\cos\left(\frac{\pi n}{2}\right)\right).
\]

\begin{enumerate}

\item Evaluate
\[ 
\sum_{k=0}^n \binom{4n}{4k}.
\]

\item {\bf Not Required/Extra Credit}: Use the Binomial Theorem to evaluate 
\[ 
\sum_{k=0}^n \binom{3n}{3k}.
\]

{\bf Hint}: In lecture we proved a similar identity by letting $x=1$ and $y= \sqrt{-1}$ in the Binomial Theorem.  What happens when you try something similar using each of the three complex solutions to $z^3 = 1$?

\end{enumerate}



\item In lecture we defined $\binom{n}{k_1,k_2, \ldots, k_t}$ when $k_1+\cdots + k_t = n$ and each $k_i$ is an integer.  Recall that if at least one of these $k_i$ is less than zero, then this multinomial coefficient is zero.  This is often useful.  For example it makes the identity $\binom{n}{k} = \binom{n-1}{k-1} + \binom{n-1}{k}$ true even when $k=0$.

\begin{enumerate}
\item Suppose $n \ge 1, k_1+k_2+\cdots +k_t = n$ and each $k_i \ge 0$.  Prove the identity
\begin{eqnarray*}
\binom{n}{k_1,k_2, \ldots, k_t}  & = & \binom{n-1}{k_1-1,k_2, \ldots, k_t} \\
&  & +  \binom{n-1}{k_1,k_2-1, \ldots, k_t} \\
&  & + \cdots +  \binom{n-1}{k_1,k_2, \ldots, k_t-1}.
\end{eqnarray*}
Can you think of a combinatorial proof?

\item Use part (a) to prove the multinomial theorem by induction.  That is, prove that for any nonnegative integer $n$,
\[
(x_1 + \cdots + x_t)^n = \sum_{k_1+\cdots + k_t = n \atop \text{and each } k_i \ge 0} \binom{n}{k_1,k_2, \ldots, k_t} x_1^{k_1} x_2^{k_2}\cdots x_t^{k_t}.
\]

\end{enumerate}

\item Prove the following identity: If $m$ and $n$ are nonnegative integers and $a+b+c = m+n$, then
\[
\sum_{\alpha+\beta+\gamma = m} \binom{m}{\alpha, \beta, \gamma} \binom{n}{a-\alpha, b - \beta, c- \gamma} = \binom{m+n}{a,b,c}.
\]
Recall that a multinomial coefficient $\binom{n}{a,b,c} = 0$ if any of $a,b,c$ are negative.




\item You are going to play paintball with $5$ of your friends.  When you are getting ready to split up into teams, $4$ local ninjas decide to play with you.  As everyone knows, ninjas are excellent at paintball.  You do not want to play against a team with four ninjas on it.  
\begin{enumerate}
\item How many ways are there to split up the $10$ of you, (you, your $5$ friends, and the $4$ ninjas)  into two teams of $5$, so that your team has at least one ninja?  
\item How many ways are there so that both teams have at least one ninja?
\end{enumerate}

%\item 
%\begin{enumerate}
%\item How many ways are there to line up $4$ boys and $8$ girls so that no two boys are next to each other?  

%{\bf Note}: Each child is a unique, so $B_1 B_2 B_3 B_4 G_1 G_2 \cdots G_8$ is a different lineup than $B_2 B_1 B_3 B_4 G_1 G_2 \cdots G_8$, (although neither one counts for this question.)

%\item How many ways are there to seat these $4$ boys and $8$ girls at a circular table (with labeled chairs) so that no two boys are next to each other?
%\end{enumerate}

%\item How many pairs of subsets $A, B$ of $\{1,2,\ldots, n\}$ are there such that $A$ is contained in $B$?

\end{enumerate}



\end{document}