\documentclass[11pt]{article}
\setlength{\oddsidemargin}{0in}
\setlength{\evensidemargin}{0in}
\setlength{\textwidth}{6.5in}
\setlength{\parindent}{0in}
\setlength{\parskip}{\baselineskip}

\usepackage{url}

\usepackage{amsmath}
\usepackage{amsthm}
\usepackage{amssymb}
\usepackage[dvips]{graphics}
\usepackage{epsfig}
\usepackage{color}
\usepackage{fullpage}
\newcommand{\F}{{\mathbb F}}

\begin{document}
\begin{center}
{\Large \bf Math 175: Combinatorics} \\
{\Large \bf Homework 3: Solutions}\\
\end{center}

\vspace{5mm}

\centerline{ \bf \Large Problems}


\begin{enumerate}

\item We saw in lecture that if there are two possible birthdays $A$ and $B$  that occur with probabilities $p$ and $1-p$, the the probability that two people have a shared birthday is $2 p(1-p)$.  This is a function of a single variable $p$ that reaches a maximum at $p = \frac{1}{2}$.  We can see this by taking the derivative with respect to $p$ to find a critical point and checking that it gives a maximum.

Suppose there are three possible birthdays: $A,B,C$ and they occur with probabilities $p_1, p_2$, and $p_3$, respectively.  A special case of what we saw in class is that the probability that two people do not share a birthday is given by
\[
f(p_1, p_2, p_3) = 2\left(p_1 p_2 + p_1 p_3 + p_2 p_3\right).
\]
\begin{enumerate}
\item Suppose that $p_3$ is fixed.  We see that $p_1 + p_2 = 1-p_3$, so $p_2 = (1-p_3) - p_1$.  Find the value of $p_1$ that maximizes the function \emph{of a single variable}
\[
f(p_1) = 2 \left(p_1 ((1-p_3) - p_1) + p_1 p_3 + ((1-p_3) - p_1) p_3\right).
\]


{\bf Solution}:  We see that 
\[
f'(p_1) = -2 p_1 + (1- p_3),
\]
so $f'(p_1) = 0$ implies that $p_1 = p_2 = \frac{1-p_3}{2}$.  By taking a second derivative we see that this is a maximum.  For any fixed $p_3$ the function increases as $p_1$ and $p_2$ come together, so this gives the absolute maximum.



\item When there are $n$ possible birthdays with probabilities $p_1,p_2,\ldots, p_n$, the probability that two people do not have a shared birthday is given by
\[
f(p_1,\ldots, p_n) = 2 \sum_{1\le i < j \le n} p_i p_j.
\]

Suppose that $p_3,\ldots, p_n$ are fixed.  As above, we can express $p_2$ in terms of $p_1$ and these fixed constants.  Find the value of $p_1$ that maximizes the function \emph{of a single variable}
\[
f(p_1) = 2 \sum_{1\le i < j \le n} p_i p_j.
\]


{\bf Solution}: We see that 
\[
f'(p_1) = -2 p_1 + (1-p_3 - p_4-\cdots - p_n),
\]
so $f'(p_1) = 0$ implies that $p_1 = p_2 = \frac{1-p_3-\cdots - p_n}{2}$.  Again, taking a second derivative shows that this is a maximum.

For any fixed $p_3,\ldots, p_n$ the function increases as $p_1$ and $p_2$ come together, so this gives the absolute maximum.


\item Using the previous result, show that the uniform distribution, where each $p_i = \frac{1}{n}$ maximizes the probability that our two people do not share the same birthday.


{\bf Solution}:  The fact that we used $p_1$ and $p_2$ in the two steps above is not so important.  Given any $(i,j)$ with $1\le i < j \le n$, we can think of every other $p_k$ being fixed and then write $p_j$ as a linear function of $p_i$.  We now think of $f(p_i)$ as a function of a single variable, in fact, it gives a concave-down parabola.  Taking a derivative shows that this function increases as $p_i$ and $p_j$ come together.

Now suppose that there is some set of values $(p_1,p_2,\ldots, p_n)$ maximizing $f(p_1,\ldots, p_n)$.  We claim that $p_1 = p_2 = \cdots = p_n = \frac{1}{n}$.

We argue by contradiction.  Suppose that the maximum value of $f(p_1,\ldots, p_n)$ is not given by the case where each $p_i$ is equal.  So, for this $(a_1,\ldots, a_n)$ giving the maximum value, there exists a pair $(j,k)$ with $1\le j < k \le n$ and $a_j \neq a_k$.  Part (b) shows that if we make  $a_j$ and $a_k$ ``come together'', then the value of $f(p_1,\ldots, p_n)$ increases, contradicting the assumption that $(a_1,a_2,\ldots, a_n)$ is where the function is maximized.  Therefore, the maximum value does come from the case where each $p_i = \frac{1}{n}$.



\end{enumerate}

\item {\bf Exercise 2.5.1 of LVP}: What is the following sum?
\[
\frac{1}{1\cdot 2}+\frac{1}{2\cdot 3} +\frac{1}{3\cdot 4} + \cdots + \frac{1}{(n-1)\cdot n}.
\]
Experiment, conjecture the value, and then prove it by induction.

{\bf Solution}:  We can try out the first few examples and see if we notice a nice pattern.  For $n=2$ we get $\frac{1}{2}$. For $n=3$ we get $\frac{2}{3}$. For $n = 4$ we get $\frac{3}{4}$.  Our induction hypothesis is that 
\[
\sum_{k=2}^n \frac{1}{(k-1)k} = \frac{n-1}{n},
\]
and our goal is to show that 
\[
\sum_{k=2}^{n+1} \frac{1}{(k-1)k} = \frac{k}{k+1}.
\]
By the induction hypothesis, this sum is given by
\[
\frac{n-1}{n} + \frac{1}{n(n+1)} = \frac{n^2-1 + 1}{n(n+1)} = \frac{n}{n+1},
\]
completing the proof.


\item {\bf Exercise 2.5.2 of LVP}: What is the following sum?
\[
0\cdot \binom{n}{0} + 1 \cdot \binom{n}{1} + 2\cdot \binom{n}{2} + \cdots + (n-1) \binom{n}{n-1} + n \cdot \binom{n}{n}.
\]
Experiment, conjecture the value, and then prove it by induction.  (You many want to try to prove this result by induction and also by combinatorial arguments.)


{\bf Solution}: We can try out the first few examples and see if we notice a nice pattern.  For $n=0$ this is $0$. For $n=1$ this is $1$.  For $n=2$ this is $4$.  For $n=3$ this is $12$.  For $n=4$ this is $32$.  At this point you might conjecture that 
\[
\sum_{k=0}^n k \binom{n}{k} = n \cdot 2^{n-1}.
\]  
This will be our induction hypothesis.  Our goal is to show that 
\[
\sum_{k=0}^{n+1} k \binom{n+1}{k} = (n+1) \cdot 2^n.
\]
This sum is equal to 
\[
\sum_{k=0}^{n+1} k \left( \binom{n}{k} + \binom{n}{k-1}\right) = \left(\sum_{k=0}^n k \binom{n}{k} \right) + \left( \sum_{k=0}^n (k+1) \binom{n}{k}\right).
\]
where we note that $\binom{n}{-1} = \binom{n}{n+1} = 0$.  By the induction hypothesis, this is equal to
\[
2 \cdot (n  2^{n-1}) + \sum_{k=0}^{n} \binom{n}{k} = n\cdot 2^n + 2^n = (n+1) 2^n,
\]
completing the proof.

{\bf Solution 2}:  Here we give an algebraic interpretation for this result.  Recall that $\binom{n}{k} = \binom{n}{n-k}$.  That means 
\[
k \binom{n}{k} + (n-k) \binom{n}{n-k} = n \binom{n}{k}.
\]
We are grouping together the terms $\binom{n}{0}$ and $\binom{n}{n},\ \binom{n}{1}$ and $\binom{n}{n-1}$, and so on.  If $n$ is even, this sum is then
\[
\frac{n}{2} \cdot \binom{n}{n/2} + n \cdot \sum_{k=0}^{n/2-1} \binom{n}{k},
\]
which is equal to 
\[
\frac{n\cdot 2^n - n \cdot \binom{n}{n/2}}{2} + \frac{n}{2} \binom{n}{n/2} = n \cdot 2^{n-1}.
\]

If $n$ is odd, then this is equal to 
\[
n \cdot \sum_{k=0}^{\frac{n-1}{2}}  \binom{n}{k} = n \cdot 2^{n-1},
\]
completing this proof.

\item In order to get a feel for how induction arguments work, it is instructive to see a bunch of convincing looking induction arguments that are actually wrong.  Read Exercises 2.1.12 and 2.1.13 in the LVP book.  There are solutions for both of these in the back of the book.  


Identify and explain the flaw in the following two induction `proofs'.
\begin{enumerate}
\item CLAIM: For all positive integers $n$,
\[ 
\sum_{j=1}^n j = \frac{1}{2} \left(n+\frac{1}{2}\right)^2.
\]
We check that this holds for $n=1$.  We suppose that this is true for $n$ and prove it for $n+1$.  We have
\begin{eqnarray*} 
\sum_{j=1}^{n+1} j & = &  \sum_{j=1}^n j + (n+1) \\
& = & \frac{1}{2} \left(n+\frac{1}{2}\right)^2 + (n+1)\ \text{ by the induction hypothesis} \\
& = & \frac{1}{2}\left(n^2 + n +\frac{1}{4} + 2n+2\right) \\
& = & = \frac{1}{2} \left( (n+1) + \frac{1}{2} \right)^2.
\end{eqnarray*}
This is what we wanted to show.  Therefore this statement holds for all positive integers $n$.

{\bf Solution}: This argument is correct except that the base case $n=1$ does not actually hold.  The thing to take away from this example is that you really have to check the base case when doing a proof by induction.

\item Claim: All positive integers are equal.  To prove this claim, we will prove by induction that for all positive integers $n$ the following statement holds.
\[
\text{For any positive integers } x \text{ and } y \text{, if } \max(x,y) = n \text{ , then } x=y.
\]
(Here $\max(x,y)$ denote the larger of the two numbers $x$ and $y$, or the common value if both are equal.)  We call this statement $P(n)$.  Clearly $P(1)$ is true, since $\max(x,y) = 1$ forces $x=1$ and $y=1$, so $x=y$.

Suppose that $P(n)$ is true.  We will prove that $P(n+1)$ is true.  Let $x,y$ be positive integers such that $\max(x,y) = n+1$.  Then 
\[
\max(x-1,y-1) = \max(x,y) -1 = n+1-1 = n.
\]
By the induction hypothesis, it follows that $x-1 = y-1$, and therefore $x=y$.  This proves $P(n+1)$ holds.  We now conclude that $P(n)$ holds for all positive integers $n$, so $\max(1,n) = n$ for any positive integer $n$. Therefore $1=n$ for any positive integer $n$.

{\bf Solution}: We notice that $P(1)$ is true but that $P(2)$ is false, so something must have gone wrong in the induction step.  Suppose $x,y$ are positive integers such that $\max(x,y) = 2$.  It is true that $\max(x-1,y-1) = \max(x,y) - 1 = 1$, but we are not able to apply $P(1)$ to $x-1$ and $y-1$.  Why not?  In order to apply $P(1)$ we need that $x-1$ and $y-1$ are both positive integers.  If either of $x,y$ is actually equal to $1$, then at least one of these is not a positive integer, so we cannot apply $P(1)$.  Therefore, there is a flaw in the induction step of this argument and the conclusion does not hold.  The moral of the story is that when you apply the induction step of an argument, you really have to check that what you're applying it to satisfies the necessary conditions of your induction hypothesis.



\end{enumerate}



\item  In class we saw that the sum of the first $n$ positive integers is $\frac{n(n+1)}{2}$.  Exercise 2.1.8 in LVP, which has a solution in the back, shows that $\sum_{j=1}^n  j^2 = \frac{n(n+1)(2n+1)}{6}$. 
\begin{enumerate}
\item Prove by induction that 
\[
\sum_{j=1}^n j^3 = \left( \sum_{j=1}^n j \right)^2 = \frac{n^2 (n+1)^2}{4}.
\]


{\bf Solution}:  We check that this is true for $n=1$, since $1^3 = \frac{1^2\cdot 2^2}{4}$.  We suppose that this holds for $n$ and use it to prove the case $n+1$.  We have
\begin{eqnarray*}
\sum_{j=1}^{n+1} j^3 & = & (n+1)^3 + \sum_{j=1}^n j^3 = \frac{4(n+1)^3 + n^2(n+1)^2}{4}  \\
& = & \frac{(n+1)^2(4n+4 + n^2)}{4} = \frac{(n+1)^2 (n+2)^2}{4},
\end{eqnarray*}
which completes the proof.




\item Prove using induction that 
\[
\sum_{j=1}^n (-1)^{j-1} j^2 = (-1)^{n-1} \left(\frac{n(n+1)}{2}\right).
\]


{\bf Solution}:  We check that this is true for $n=1$, since $1^2 = (-1)^0\cdot \frac{1\cdot 2}{2}$.  We suppose that this holds for $n$ and show that it holds for $n+1$. We have
\begin{eqnarray*}
\sum_{j=1}^{n+1} (-1)^{j-1} j^2 & = & (-1)^n (n+1)^2 + \sum_{j=1}^{n} (-1)^{j-1} j^2 = (-1)^n (n+1)^2 +  (-1)^{n-1} \left(\frac{n(n+1)}{2}\right)  \\
& = & (-1)^n  \frac{(n+1)(n+2)}{2},
\end{eqnarray*}
which completes the proof.



\end{enumerate}

\item 
\begin{enumerate} 
\item Use the Hockey-Stick Identity from lecture, equation (3.5) on page 53 of the LVP book, to prove the following:
\[ 
\binom{r}{r} + \binom{r+1}{r} + \cdots + \binom{n}{r} = \binom{n+1}{r+1}.
\]


{\bf Solution}: We first recall the hockey-stick identity:
\[
\sum_{k=0}^{r'} \binom{n'+k}{k} = \binom{n'+r'+1}{r'}.
\]
If this holds for any values of $n'$ and $r'$, then it holds for $n' = r$ and $r' = n-r$.  This gives
\begin{eqnarray*}
\sum_{k=0}^{n-r} \binom{r+k}{k} & = & \sum_{k=0}^{n-r} \binom{r+k}{r} \\
& = & \binom{n'+r'+1}{r'} = \binom{n+1}{n-r} = \binom{n+1}{r+1},
\end{eqnarray*}
completing the proof.

We note that what's really going on here is that we're taking the hockey-stick identity and noting that if we replace each $\binom{n}{k}$ with $\binom{n}{n-k}$ we are flipping each coefficient over the vertical line down the middle of Pascal's triangle.  The tricky part comes in choosing the right substitution to write this down as a nice sum.


\item In lecture we gave a combinatorial proof of the Hockey-Stick identity.  Give another.  It's fine to give a combinatorial proof of the identity from part (a) of this problem, rather than of the Hockey-Stick result directly. It will be helpful to look at Exercise 3.6.4 and its Hint in the back of the LVP book.

{\bf Solution}: Suppose we pick $r+1$ items out of the set $\{1,2,\ldots, n+1\}$.  The largest of these $r+1$ items can be any element of $\{r+1,r+2,\ldots, n+1\}$.  If the largest element is $r+1+k$, then out of the first $r+k$ elements, we chose $r$ of them.  So, there are $\binom{r}{r}$ ways for $r+1$ to be the largest element chosen, $\binom{r+1}{r}$ ways for $r+2$ to be the largest, and so on, up to $\binom{n}{r}$ ways for $n+1$ to be the largest.  Adding these gives a combinatorial proof of the identity.

\end{enumerate}



\item For this problem, imagine we are flipping a fair coin.  That is, on each flip there is a $50\%$ chance of getting H and a $50\%$ chance of getting T.  We have already seen that there are $2^n$ total outcomes when flipping $n$ times, since this is the same as a word of length $n$ made up of H's and T's.  These questions look like they are about probability, but they can be interpreted as questions about counting words from a two-letter alphabet with certain properties.
\begin{enumerate}
\item Flip a coin $2n$ times.  What is the probability of getting either $n-1$ heads or $n$ heads?


{\bf Solution}: There are $2^{2n}$ total possible sequences of coin flips.  There are $\binom{2n}{n-1}$ ways to get $n-1$ heads exactly and $\binom{2n}{n}$ ways to get $n$ heads exactly.  Therefore this probability is
\[
\frac{\binom{2n}{n-1}+\binom{2n}{n}}{2^{2n}}.
\]


\item Suppose you flip a coin $2n+1$ times.  What is the probability of getting either $n$ heads or $n+1$ heads? 

{\bf Solution}: There are $2^{2n+1}$ total sequences of coin flips.  There are $\binom{2n+1}{n+1}$ ways to get $n+1$ heads exactly and $\binom{2n+1}{n}$ ways to get $n$ heads exactly.  Therefore this probability is
\[
\frac{\binom{2n+1}{n+1}+\binom{2n+1}{n}}{2^{2n+1}}.
\]

\item Show that your answers to parts (a) and (b) are equal.  You can do this using a binomial identity from lecture, or (and this solution is nicer), you can give a combinatorial interpretation of what's going on here.


{\bf Solution}: We have to show that 
\[
2 \left(\binom{2n}{n-1}+\binom{2n}{n}\right) = \binom{2n+1}{n+1}+\binom{2n+1}{n}.
\]
We first note that $\binom{2n}{n-1} = \binom{2n}{n+1}$.  Since $\binom{2n}{n+1} + \binom{2n}{n} = \binom{2n+1}{n+1}$ the left hand side is 
\[ 
2 \binom{2n+1}{n+1} = \binom{2n+1}{n}+\binom{2n+1}{n+1},
\]
completing the proof.  

We could also argue as follows.  Suppose we have flipped a coin $2n$ times.  We then flip it one more time.  If the total number of heads is now $n$ or $n+1$, then the total number of heads in the first $2n$ must have been $n-1$ with the last flip being a head, $n$ with the last flip being either a head or a tails, or $n+1$ with the last flip being tails.  Therefore, the probability of getting $n+1$ or $n$ heads after $2n+1$ flips is half of the probability of getting $n-1$ heads in $2n$ flips, plus the probability of getting $n$ heads in $2n$ flips, plus half the probability of getting $n+1$ heads in $2n$ flips.  These first and last probabilities are the same, so the sum of these three terms is the probability of getting $n-1$ heads in $2n$ flips plus the probability of getting $n$ heads in $2n$ flips.  This completes the second proof.
\end{enumerate}



\end{enumerate}



\end{document}