\documentclass[11pt]{article}
\setlength{\oddsidemargin}{0in}
\setlength{\evensidemargin}{0in}
\setlength{\textwidth}{6.5in}
\setlength{\parindent}{0in}
\setlength{\parskip}{\baselineskip}

\usepackage{amsmath}
\usepackage{amsthm}
\usepackage{amssymb}
\usepackage[dvips]{graphics}
\usepackage{epsfig}
\usepackage{color}
\usepackage{fullpage}
\newcommand{\F}{{\mathbb F}}

\begin{document}
\begin{center}
{\Large \bf Math 175: Combinatorics} \\
{\Large \bf Homework 1: Solutions}\\
\end{center}

\vspace{5mm}

\begin{enumerate}

\item \textbf{Exercise 2.1.3 in HHM}: There are $30$ teams in the National Basketball Association: $15$ in the Western Conference, and $15$ in the Eastern Conference.
\begin{enumerate}

\item Suppose each of the teams in the league has one pick in the first round of the NBA draft.  How many ways are there to arrange the order of the teams selecting in the first round of the draft?

{\bf Solution}: Since each of the $30$ teams picks exactly once in the first round, this question is the same as asking how many ways there are to rearrange $30$ objects.  This is $30!$.

\item Suppose that each of the first three positions in the draft must be awarded to one of the fourteen teams that did not advance to the playoffs that year.  How many ways are there to assign the first three positions in the draft?

{\bf Solution}: We have to choose the first $3$ teams: There are $14$ choices for the first team to pick, $13$ choices for the second team to pick, and $12$ choices for the third team to pick.  Therefore, there are $14\cdot 13\cdot12$ total ways to assign the first three picks.

\item How many ways are there for eight teams from each conference to advance to the playoffs, if order is unimportant?

{\bf Solution}: There are $\binom{15}{8}$ subsets of $8$ teams from the $15$ teams of the Eastern Conference, and $\binom{15}{8}$ subsets of $8$ teams from the Western Conference.  Therefore, the total number of ways for $8$ teams from each conference to be chosen for the playoffs is $\binom{15}{8}^2$.
 
\item Suppose that every teams has three centers, four guards, and five forwards.  How many ways are there to select an all-star team with the same composition from the Western Conference?

{\bf Solution}: We must choose a team of $3$ centers, $4$ guards, and $5$ forwards, out of a total pool of $3\cdot 15 = 45$ centers, $4\cdot 15 = 60$ guards, and $5\cdot 15 = 75$ forwards.  The number of ways of doing this is $\binom{45}{3}\cdot \binom{60}{4} \cdot \binom{75}{5}$.

\end{enumerate}

\item How many submultisets of $\{1,1,2,2,\ldots, n, n\}$ have size that is divisible by $3$?  \\
For example, when $n = 2$ we only want to count the submultisets of size $0$ and $3$.

{\bf Solution}: By testing small values of $n$ you may guess that the formula is $3^{n-1}$.  We will prove that this is correct.

We know from lecture that there are $3^n$ total submultisets of $\{1,1,2,2,\ldots, n, n\}$.  Every such multiset has size that is either divisible by $3$, one more than a multiple of $3$, or two more than a multiple of $3$.  We will show that this gives a division of the entire collection of submultisets into three sets of equal size.  

Let $A$ be the set of submultisets of $\{1,1,2,2,\ldots,n,n\}$ of size divisible by $3$.\\
Let $B$ be the set of submultisets of $\{1,1,2,2,\ldots,n,n\}$ of size one more than a multiple of~$3$.\\
Let $C$ be the set of submultisets of $\{1,1,2,2,\ldots,n,n\}$ of size two more than a multiple of~$3$.

We will show that for any $n$ the sets $A,B$, and $C$ each have the same size by showing that there is a bijection from $A$ to $B$ and that there is a bijection from $A$ to $C$.  This shows that we have divided the $3^n$ total submultisets into three sets of equal size, so each set has size $3^{n-1}$.

We first find a bijection from $A$ to $B$.  Given a multiset $x$ in $A$, if the element $n$ appears in $A$ zero or one time, we let $f(x)$ be the multiset that is exactly the same as $x$ but has one more $n$ in it.  If the element $n$ appears in $x$ twice, we let $f(x)$ be the multiset that is the same as $x$ except that it has no copies of $n$.  We see that this function $f$ takes any multiset in $A$ to a multiset in $B$.

We similarly define a bijection from $A$ to $C$. Given a multiset $x$ in $A$, if the element $n$ appears in $A$ zero times, we let $g(x)$ be the multiset that is exactly the same as $x$ but has two more copies of $n$ in it.  If the element $n$ appears in $x$ once or twice, we let $g(x)$ be the multiset that is the same as $x$ except that it has one fewer copy of $n$.  We see that this function $f$ takes any multiset in $A$ to a multiset in $C$.

We now check that $f$ and $g$ are both bijections, that each map gives a `one-to-one' correspondence.  Given a multiset $y$ in $B$ we want to show that there is exactly one $x$ in $A$ with $f(x) = y$.  If $y$ has one or two copies of $n$, then the set that is exactly the same as $y$ but with one fewer copy of $n$ is sent to $y$.  So, $f$ is a surjection.

It is not possible for two different multisets in $A$ to be sent to the same element of $B$ by $f$.  If it were, then because all $f$ does is change the number of copies of $n$ that a multiset has, these two multisets would have to have the same number of copies of $1$, of $2$, and so on, up to $n-1$.  Once we have fixed the number of copies of each of these numbers that a multiset has, there is exactly one choice of zero, one, or two copies of $n$ that gives a set of size divisible by $3$.  So it is not possible for these two sets to be different.  Therefore, $f$ is an injection and is therefore a bijection.  The argument that $g$ is a bijection is almost exactly the same.

{\bf Solution 2}: We can choose a submultiset of $\{1,1,2,2,\ldots, n,n\}$ in two parts.  First, we choose a submultiset of $\{1,1,2,2,\ldots, n-1,n-1\}$ and then we choose how many copies of $n$ to include: $0,1$, or $2$.  

Let $S$ be our chosen submultiset of $\{1,1,2,2,\ldots, n-1,n-1\}$.  Suppose the size of $S$ is $m$.  No matter what the value of $m$ is, there is a unique choice of how many copies of $n$ to include that makes the total size of the submultiset divisible by $3$.  This is equivalent to saying that for any positive integer $m$ there is a unique element $x \in \{0,1,2\}$ so that $3$ divides $x+m$.  This is clear (if $m \equiv 0 \pmod{3}$, choose $x = 0$, for example).

\item \textbf{Exercise 2.1.11 in HHM}: Suppose a positive integer $N$ factors as $N = p_1^{a_1}p_2^{a_2}\cdots p_m^{a_m}$, where $p_1,p_2,\ldots, p_m$ are distinct prime numbers and $a_1, a_2,\ldots, a_m$ are all positive integers.  How many different positive integers are divisors of $N$?

{\bf Solution}: Suppose claim that there is a bijection between factors of $N$ and submultisets of $\{p_1,\ldots, p_1, p_2,\ldots, p_2,\ldots, p_m,\ldots, p_m\}$ where $p_i$ occurs $a_i$ times.  That is, in order to describe a factor, we have to choose how many $p_1$'s divide this factor, how many $p_2$s divide this factor, and so on.  We have $a_1+1$ choices for the number of $p_1$s since we can pick any number from $0$ to $a_1$.  Similarly, we have $a_i+1$ choices for the number of $p_i$'s for each $i$.  Since these choices are independent, the total number of factors is $\prod_{i=1}^k (a_i+1)$.


\item There is an infinite line of houses with doors numbered $1,2,3,\ldots$ and so on.  All of the doors begin open.  A neighborhood resident comes by and closes every door.  Then another person comes by and opens every other door, that is, opens doors $2,4,6,\ldots$ and so on.  A third person comes by and changes the status of every third door, that is, if it's open she closes it, and if it's closed she opens it.  For example, since door 3 is now closed, she opens it.  Since door 6 is open, she closes it.  Then a fourth person comes by and does the analogous thing, then a fifth, and so on.  Give a simple description of the doors that will be closed in this process.

\textbf{Hint:} It might be helpful to consider what happens for the first $10$ doors or so.  For example, once door $1$ is closed by the first person, it is never opened again, since the $n$th person in the process starts by going to door $n$.  The second door, is first closed by person one, and then opened by person two, and remains open after that.

\textbf{Hint:} Use previous problem. 

{\bf Solution}: We first note that among the first $10$ doors, only doors $1,4,9$ remain closed after the first $10$ people have gone.  We will prove that the only doors left closed by this process are given by the squares.  

First we note that the $m$\textsuperscript{th} person in this process changes the status of door number $n$ if and only if $m$ divides $n$.  Therefore, given a number $n$ we have to determine how many factors it has.  If $n$ has an odd number of factors, then door $n$ remains closed, and if $n$ has an even number of factors, then door $n$ remains open.  

We saw in the last problem that if the prime factorization of $n$ is given by $p_1^{a_1} p_2^{a_2} \cdots p_k^{a_k}$ then the number of factors of $n$ is $\prod_{i=1}^k (a_i+1)$.

When is this product odd?  A product of positive integers is odd if and only if each one is odd.  So this product is odd if and only if each $a_i$ is even.  But, each $a_i$ is even if and only if $n$ is a perfect square since the square root of $n$ is $p_1^{a_1/2} p_2^{a_2/2}\cdots p_k^{a_k/2}$.  Therefore, a number has an odd number of factors if and only if it is a perfect square, and door number $n$ stays closed if and only if $n$ is a perfect square.

\item \textbf{Exercise 2.1.12 in HHM}: Assume that a positive integer cannot have $0$ as its leading digit.
\begin{enumerate}
\item How many five-digit positive integers have no repeated digits at all?

{\bf Solution}:  We have $9$ choices for the first digit since it cannot be a $0$.  No matter what the first digit is, we have $9$ remaining choices for the second digit, since it must be different from the first.  We then have $8$ choices for the third, $7$ for the fourth, and $6$ for this fifth.  This gives $9\cdot 9 \cdot 8\cdot 7\cdot 6$ total five-digit numbers with no repeated digits.

\item How many have no consecutive repeated digits?

{\bf Solution}: We can choose our digits one at a time.  There are $9$ choices for the first digit.  No matter what the first digit is, there are $9$ choices for a different second digit.  We then must pick a third digit distinct from our second digit, and again there are $9$ choices.  We similarly have $9$ choices for the fourth and fifth digits.  This gives $9^5$ total five-digit numbers.


\item How many have at least one run of consecutive repeated digits (for example, $23324, 45551$, or $15155$, but not $12121$)?

{\bf Solution}: There are $9 \cdot 10^4$ total five-digit numbers, since we have $9$ choices for the first digit and then $10$ choices for each other digit.  In the previous part we counted the number of these that have no repeated digits.  By subtracting, we see that the number of five-digit numbers with at least one repeated digit is $9\cdot 10^4 - 9^5$.

\end{enumerate}

\item \textbf{Exercise 1.8.29 in LVP}: What is the number of ways to color $n$ objects with $3$ colors if every color must be used at least once?

\textbf{Hint:} Consider the number of ways so that at least one color is not used.

{\bf Solution}: This problem introduces a new idea.  We know that the total number of ways of coloring $n$ objects with $3$ colors is $3^n$.  We want to subtract the number of ways so that at least one color is not used.  Let us suppose that our $3$ colors are called $A,B$, and $C$.  How many colorings only use $A$ and $B$?  This is $2^n$.  So the number that only use $A$ and $B$ plus the number that only use $A$ and $C$ plus the number that only use $B$ and $C$ is $3 \cdot 2^n$.  

However, the coloring where we color every object with $A$ is counted twice- it's counted once in the set of colorings that only use $A$ and $B$ and once in the set of colorings that only use $B$ and $C$.  Therefore, we must add a $1$ for this coloring that is subtracted twice.  Similarly, we add in $1$ for the all $B$ coloring and $1$ for the all $C$ coloring.  This means that the number of colorings where there is at least one coloring not used is $3\cdot 2^n -3$.

We see that the total number of colorings where each color is used at least once is 
\[
3^n - (3\cdot  2^n -3) = 3^n - 3\cdot 2^n + 3.
\]  
It may be helpful to check that this formula is correct for $n=1,2$, and $3$.  In the first two cases there are no colorings where each color is used at least once, and for $n=3$ there are exactly $6$.  This matches the formula given here.


\item \textbf{Exercise 1.8.27 in LVP}:  Alice has $10$ balls (all different).  First, she splits them into two piles; then she picks one of the piles with at least two elements, and splits it into two; she repeats this until each pile only has one element.

\begin{enumerate}
\item How many steps does this take?

{\bf Solution}:  At the end of the process there are $10$ piles each with one element.  Every time Alice divides the balls she increases the number of piles by one.  Therefore, this process takes exactly $9$ divisions.


\item Show that the number of different ways in which she can carry out this procedure is 
\[ \binom{10}{2}\cdot \binom{9}{2} \cdots \binom{3}{2} \cdot \binom{2}{2}.\]

\textbf{Hint}: Imagine this procedure backward.

{\bf Solution}: To count the number of ways to carry out this process it helps to think of this process going in reverse.  Instead of dividing piles starting from the beginning, we consider Alice's moves in reverse, combining two piles at each step, starting from the very end.  Here, there are $10$ piles and she has to choose to combine $2$ of them.  In the previous step there are $9$ piles and she has to choose two of them.  This pattern continues until there are just two piles left to combine, giving $\binom{2}{2}$ choices.  The choice that Alice makes at each step doesn't change the total number of choices she can make at the next step, so the total number of ways to carry out this process is given by the product
\[ 
\binom{10}{2}\cdot \binom{9}{2} \cdots \binom{3}{2} \cdot \binom{2}{2}.
\]

\end{enumerate}



\end{enumerate}


\end{document}