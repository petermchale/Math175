\documentclass[11pt]{article}
\setlength{\oddsidemargin}{0in}
\setlength{\evensidemargin}{0in}
\setlength{\textwidth}{7in}
\setlength{\parindent}{0in}
\setlength{\parskip}{\baselineskip}

\usepackage{amsmath}
\usepackage{amsthm}
\usepackage{amssymb}
\usepackage[dvips]{graphics}
\usepackage{epsfig}
\usepackage{color}
\usepackage{url}

\usepackage{fullpage}
\newcommand{\F}{{\mathbb F}}

\begin{document}
\begin{center}
{\Large \bf Math 175: Combinatorics} \\

\vspace{3mm}

\bf Winter 2018 Course Information and Syllabus\\
\vspace{3mm}
\end{center}

\vspace{3mm}

{\Large \centerline{Course Goals}}
Combinatorics can be described as the art of counting.  The subject is built up from simple concepts but we will quickly run into difficult problems that require clever insights.  We will learn what sorts of objects mathematicians are interested in counting and many techniques for studying them.

Another goal, probably more important than learning a body of material, is to develop your mathematical reasoning ability.  We will learn lots of problem-solving techniques in this course and use them to solve lots of different kinds of problems.  We will also do lots of proofs.

This course will serve as a good foundation for aspiring mathematicians, but will also be very useful even if you never intend to take another math class.  Ideas we will encounter  have far-ranging applications in computer science, applied mathematics, and other quantitative areas.  The experience you will gain here will help prepare you to reason clearly in our increasingly quantitative and data-driven world.

{\Large \centerline{Grading}}
\begin{itemize}
\item Homework: 25\%
\item Quizzes in Discussion Section (there will be approximately four): 10\%
\item First Midterm Exam (in class, Fri 2 Feb): 15\%
\item Second Midterm Exam (in class, Wed 28 Feb): 15\%
\item Final Exam (Wed 21 March, 1:30 - 3:30): 35\%
\end{itemize}

Weekly homework will be a big part of this course.  The best way to learn any mathematical subject is by doing lots of problems, and this is especially true for combinatorics.  Some of these problems will be straightforward while others will require some very clever thinking.  I have always found that I think better about mathematics when I can discuss it with others and that I only really understand a problem when I can explain its solution to somebody else.  You are encouraged to work together on problem sets, but write up your solutions individually.  If you use outside sources (other textbooks, websites, etc.) for your homework, you must acknowledge them.

Your lowest quiz grade and lowest homework grade will be dropped.

It is expected that students will make every effort to attend exams at the scheduled times. Requests for make-ups will be ignored unless:
\begin{itemize}
\item
there is a verifiable emergency, which prevents the student from taking the exam at the scheduled time. Such emergencies include, but are not limited to, serious illness, death of immediate family member, or serious accident.
\item
there is a timely written petition for a make-up accompanied by verifiable documentary evidence.
\end{itemize}

\vspace{5 mm}


{\Large \centerline{Course Outline}}
\begin{enumerate}
\item Binomial Coefficients, Pascal's Triangle.
\item Fibonacci Numbers, Permutations and Combinations.
\item Proof by Induction, The Principle of Inclusion/Exclusion, The Pigeonhole Principle.
\item Basic Discrete Probability, Conditional Probability, The Law of Large Numbers.
\end{enumerate}

\subsubsection*{Holidays}
MLK Day Monday 15 Jan \\
Presidents' Day Monday 19 Feb 

\subsubsection*{Books}
\begin{enumerate}
{\small
\item \emph{Discrete Mathematics: Elementary and Beyond}, L. Lov\'asz, J. Pelik\'an, K. Vesztergombi. \\
ISBN: 0387955852

{\bf Note}: We plan to cover Chapters 1-5 of this book.

This book is available for free as an ebook as part of the UCI catalog: \\
\url{http://antpac.lib.uci.edu:80/record=b5799169~S7}

There is a limit to the number of users who can access the book at the same time.  If you have trouble accessing it, please wait a little while and try again.  You cannot download the entire book, but you can download individual sections (approximately 60 pages at a time).


\item \emph{Combinatoics and Graph Theory, 2nd ed.}, J. Harris, J. Hirst, M. Mossinghoff. \\ ISBN: 978-0-387-797710-6

{\bf Note}: We plan to cover Sections 2.1-2.6 of this book. 

This book is available for free to UCI students through SpringerLink:\\
\url{http://link.springer.com/book/10.1007%2F978-0-387-79711-3}
}



\end{enumerate}

\end{document}
